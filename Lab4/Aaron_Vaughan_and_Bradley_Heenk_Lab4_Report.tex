% template created by: Russell Haering. arr. Joseph Crop
\documentclass[12pt, letterpaper]{article}
\usepackage{amssymb,mathtools}
\usepackage[utf8]{inputenc}
\usepackage{setspace}
\setlength{\parskip}{1.2ex}    
\setlength{\parindent}{2em}
\usepackage[a4paper, textwidth=400pt, left=2cm]{geometry}
\renewcommand*\familydefault{\ttdefault} %% Only if the base font of the document is to be typewriter style


\begin{document}


\begin{titlepage}
    \vspace*{4cm}
    \begin{flushright}
    {\huge
        ECE 375 Lab 2\\[1cm]
    }
    {\large
        Introduction to AVR Tools
    }
    \end{flushright}
    \begin{flushleft}
    Lab Time: Friday 4-6
    \end{flushleft}
    \begin{flushright}
    Aaron Vaughan

    Bradley Heenk
    
    \vfill
    \rule{5in}{.5mm}\\
    TA Signature
    \end{flushright}

\end{titlepage}



\section{Introduction}

The purpose of this lab is to get us familiar with the process of writing our own avr assembly code. This lab uses the LCD screen. This means that we have to include the driver file in order to utilize this additional piece of hardware. The initialization of the stack pointer is crucial for the proper operation of this lab. We learn that without this initialization step, the LCD driver package runs in a continuous loop. 
\newline
The assignment is to display our name and that of our lab partner with included functionality to swap the lines that they are printed on and clear the screen with the use of the buttons connected to PORTD.
\newline
While coding this project, my lab partner and I learned how to implement for loops, if statements, and became familiarized with the formatting of the avr assembly code itself. The project reinforced our knowledge of data systems and memory management. The process is not too dissimilar from writing in C. It just takes many many more lines to orchestrate the mnemonics and produce working code.
%The lab write-up should be done in the style of a professional report/white paper.  Proper headers need to be used and written in a clean, professional style.  Proof read the report to eliminate both grammatical errors and spelling.  The introduction should be a short 1-2 paragraph section discussing what the purpose of this lab is.  This is not merely a copy from the lab handout, but rather your own personal opinion about what the object of the lab is and why you are doing it.  Basically, consider the objectives for the lab and what you learned and then briefly summarize them.  For example, a good introduction to lab 1 may be as follows.
        
                %-- Example from Lab-2 --%

% The purpose of this first lab is to provide an introduction on how to use AVRStudio4 software for this course along with connecting the AVR board to the TekBot base.  A simple pre-made “BumpBot” program was provided to practice creating a project in AVRStudio4, building the project, and then using the Universal Programmer to download the program onto the AVR board.

% The purpose of this lab is to compile a program written in c-language that will implement the bumpbot behavior. We will learn how to interact with the inputs and outputs on the ATmega128 board. I chose to write some subroutines to accomplish the turning commands and the bump command used for the challenge code. We started with the example code and cut and pasted much of the command procedures from that. We also had to look at the first lab that ws coded in avr to determine how to access the inputs (whiskers) from PORTD. It was unclear how this PORTD was connected to the physical buttons but Bradley found that by tracing out the wires on the printed board that they did indeed follow directly from the PORTD connector.

\section{Program Overview}

The avr assembly code in this lab displays the text that is reserved in data memory (hard-coded by Bradley and I) on the LCD screen. This is accomplished by reserving program memory for the string data in an array of sequentially accessible char values with a head pointer. The data is then loaded in sequentially in a char by char fashion beginning at the address \$0100. The LCD is 16 char-wide sections in length. We padded our names with spaces to fill the entire screen. The reading and writing of the char values to the LCD screen are accomplished by post-incrementing an index pointer to access the next memory address that a char needs to be written to. It goes through all the values in the Partner Name array. This repeats for the second partner's name on the second line. There are two buttons that can be pressed to initiate the writing of the strings to the LCD. One button puts name one on line one name two on line two and the other switch then swaps the names and lines. The last button (S7) clears the screen by calling the LCDClr function defined in the LCD driver package.
\newline
The details of definitions of registers, constants, subroutines and other important details follow in the next sections.

% This section provides an overview of how the assembly program works.  Take the time to write this section in a clear and concise manner.  You do not have to go into so much detail that you are simply repeating the comments that are within your program, but simply provide an overview of all the major components within your program along with how each of the components work.  Discuss each of your functions and subroutines, interesting program features such as data structures, program flows, and variables, and try to avoid nitty-gritty details.  For example, simple state that you “First initialized the stack pointer,” rather than explaining that you wrote such and such data values to each register.  These types of details should be easily found within your source code.  Also, do not hesitate to include figures when needed.  As they say, a picture is worth a thousand words, and in technical writing, this couldn’t be truer.  You may spend 2 pages explaining a function which could have been better explained through a simple program-flow chart.  As an example, the remainder of this section will provide an overview for the basic BumpBot behavior.
% The BumpBot program provides the basic behavior that allows the TekBot to react to whisker input.  The TekBot has two forward facing buttons, or whiskers, a left and a right whisker.  By default the TekBot will be moving forward until one of the whiskers are triggered.  If the left whisker is hit, then the TekBot will backup and then turn right for a bit, while a right whisker hit will backup and turn left.  After the either whisker routine completes, the TekBot resumes its forward motion.  
% Besides the standard INIT and MAIN routines within the program, three additional routines were created and used.  The HitRight and HitLeft routines provide the basic functionality for handling either a Right or Left whisker hit, respectively.  Additionally a Wait routine was created to provide an extremely accurate busy wait, allowing time for the TekBot to backup and turn.

\section{Internal Register Definitions and Constants}

In this section of code, we set up the required registers and constants. Most of the internal registers are used up in the LCD driver package so we were very restricted as to which registers we had left over to use for the implementation of our algorithm. We set the forward, reverse, and erase values corresponding to the locations of the switches which are active low.

\section{Interrupt Vectors}

There is one interrupt vector within the skeleton code for this lab. It sets up the initial starting point of our main program. Beginning at memory address \$0000 it simply jumps to the initialization routine.

\section{Program Initialization}

The stack pointer must be initialized to use the stack in the algorithm. This is typically done, as implemented in the our code, to the end of ram. It takes two cycles to perform this because the stack pointer is 16-bits but the register used to communicate with it is only 8-bits wide. Without the use of the stack pointer and use of that type of data structure, there would not be enough registers to perform the operations. This lab uses the LCD display which takes up a lot of the available registers. To use the LCD we must include some drivers and initialize the LCD display. Also, since we want to have a way to interface with the device with I/O we need to initialize PortD as an input.


%The initialization routine provides a one-time initialization of key registers that allow the BumpBot program to execute correctly.  First the Stack Pointer is initialized, allowing the proper use of function and subroutine calls.  Port B was initialized to all outputs and will be used to direct the motors.  Port D was initialized to inputs and will receive the whisker input.  Finally, the Move Forward command was sent to Port B to get the TekBot moving forward.


\section{Main Program}

The main program section of this lab runs in a loop and continuously checks for inputs from the buttons connected to PORTD. It accomplishes this by first storing the PIND values into mpr. It then compares the value of mpr to the hard coded values of the forward, reverse, and clear switches. If a switch is pressed the value on the PIND corresponding to that switch will be 1. If the pressed button matches the condition of the case, then it will send the program to that subroutine to execute the algorithm to execute that command. After a function call, we return back out and loop back up to main to start over again. If no buttons are pressed, the main program loops back and starts over.

%The Main routine executes a simple polling loop that checks to see if a whisker was hit.  This is accomplished by first reading 8-bits of data from PINE and masking the data for just the left and right whisker bits.  This data is checked to see if the right whisker is hit and if so, then it calls the HitRight routine.  The Main routine then checks to see if the left whisker is hit and if so, then it calls the HitLeft routine.  Finally a jump command is called to move the program back to the beginning of the Main Routine to repeat the process.

\section{Subroutines}

 Subroutine name:	S1\_DISPLAY \newline
 Description:	A micro-function for PARTNER\_WRITE to simplify what is said in main. This function sets everything up to display partner 1 on line 1 and partner 2 on line 2

 Subroutine name: S2\_DISPLAY \newline
 Description: A micro-function for PARTNER\_WRITE to simplify what is said in main. This function sets everything up to display partner 2 on line 1 and partner 1 on line 2

Function name: S8\_CLEAR; \newline
Description: This function clears the LCD Screen by calling the; LCDClr in the LCDDriver.asm

Subroutine name: PARTNER\_WRITE \newline
Description: Writes partner based on the line types This function depends on countin being set before being called to either 1 or a 2 to indicate which partner. The is true with count out which correlates to which line mpr is set to: 1 means line 1, 2 means line 2



% \begin{enumerate}
%     \item   \textbf{HITRIGHT ROUTINE} 
%     \newline
% The HitRight routine first moves the TekBot backwards for roughly 1 second by first sending the Move Backwards command to PORTB followed by a call to the Wait routine.  Upon returning from the Wait routine, the Turn Left command is sent to PORTB to get the TekBot to turn left and then another call to the Wait routine to have the TekBot turn left for roughly another second.  Finally, the HitRight Routine sends a Move Forward command to PORTB to get the TekBot moving forward and then returns from the routine.

% \end{enumerate}

\section{Stored Program Data}

As described in the program overview, the program data is used for a string of 8-bit char values reserved in program memory using the .db operative. We stored two 16-char values that are to be used as an array of chars to implement printing to a screen. This stored program data section is near the end of the code as an avr style convention.

\section{Additional Program Includes}

As mentioned in many other sections. The LCD requires a specific driver package. It is here, in this section that the file is included.

\section{Additional Questions}
\begin{enumerate}
    \item
    In this lab, you were required to move data between two memory types:
program memory and data memory. Explain the intended uses and key
differences of these two memory types.
\newline \newline
Program memory is 16-bits and is arranged so that we can access individual bytes using low(address) or high(address). We use it to store large data structures like arrays of chars. It's primary purpose is to store instruction words. In contrast, 8-bit data memory is used for data needed for arithmetic, logical, or other operations. These can be operands or any other useful data used in computations. If more than 8-bits are required, the data is stored sequentially.

    % 1. This lab required you to compile two C programs (one given as a sample, and another that you wrote) into a binary representation that allows them to run directly on your mega128 board. Explain some of the benefits of writing code in a language like C that can be “cross compiled”. Also, explain some of the drawbacks of writing this way.

    % The benefits of writing this in c-language is the clear advantages that a high-level language offers: we as the programmers do not need to worry about micro-operations. 
    % \newline
    % The disadvantages is that we no longer have as much control of what the CPU is actually doing. So if we cared how many clock cycles some operation requires, then using c-language to control the behavior of the device leaves us with uncertainty in this area.

    \item
    You also learned how to make function calls. Explain how making a function call works (including its connection to the stack), and explain why a RET instruction must be used to return from a function. 
\newline \newline
    A function or subroutine is called by the mnemonic rcall. When rcall is read by the compiler, the first thing that it does is push the current location in program memory onto the stack so that it can get back when the function is finished. Next, it jumps to the location in memory where the function instructions to carry out the command are stored in memory. The subroutine does its thing and then at the bottom there is a return statement. The return statement pops the return location back off the stack and uses it for the next fetch cycle. The result is that we return back out to the line where we called the function or subroutine. This way, the next line in the program can be ran.
    
    % 2. The C program you just wrote does basically the same thing as the sample assembly program you looked at in Lab 1. What is the size (in bytes) of your Lab 1 \& Lab 2 output .hex files? Can you explain why there is a size difference between these two files, even though they both perform the same BumpBot behavior?

    % My *Lab2.hex file is 958 Bytes while the *Lab1.hex file is 485 Bytes. When the c-language is translated into the machine language, there are some statements that translate into more micro-operations, even though the end result is the same. When translating from c-language to machine language there may be certain operations that just translate differently most likely due to a choice that the translation was coded so that it was easier to write.
    \newpage
    \item
    To help you understand why the stack pointer is important, comment out the stack pointer initialization at the beginning of your program, and then try running the program on your mega128 board and also in the simulator. What behavior do you observe when the stack pointer is never initialized? In detail, explain what happens (or no longer happens) and why it happens.
    \newline \newline
    Without the stack pointer the program will still compile but it it will not work on the board. I watch what happens in the debugger. When the LCDInit is called, mpr is pushed onto the stack. The consequence of this is that since the stack pointer was not initialized, it had the value of \$0000 stored as its location in memory to point to. The problem arises once the push operation is called by the LCD driver package. Push will set the register data onto the stack and decrement itself. This puts the stack pointer past the RAMEND value of \$01FF. So the stack pointer is now in a location outside of valid memory addresses and the program gets stuck in an infinate loop within a subroutine fittingly called LCDWait. 



\end{enumerate}

\section{Difficulties}

Understanding how to use the inputs from the push buttons was a bit difficult. Keeping track of all the data structures while implementing the algorithm with a limited number of registers took a little extra help from Youngbin. The extra credit was difficult to get right. We implemented it two different ways before finding the correct definition of marquee scroll in the lab4 handout.

\section{Conclusion}
The lab was fun. We learned how important the use of data structures and memory management are, as they pertain to programming a microcontroller. We were tasked with implementing an algorithm in avr assembly language that displayed text strings on an LCD display. We coded, compiled, debugged, compiled again, then loaded the program on our atmega128 boards to see it in action. Memory management and proper use of the stack was important.

%The conclusion should sum up the report along with maybe a personal though on the lab.  For example, in this lab, we were simply required to set up an AVRStudio4 project with an example program, compile this project and then download it onto our TekBot bases.  The result of this program allowed the TekBot to behave in a BumpBot fashion.  The lab was great and allowed us the time to build the TekBot with the AVR board and learn the software for this lab

\section{Source Code}

\begin{verbatim}
    

;***********************************************************
;*
;*	Aaron_Vaughan_and_Bradley_Heenk_Lab4_sourcecode.asm
;*
;*	This program uses AVR to display strings on the LCD screen
;*	S8 clears the screen, S1 Displays Bradley Heenk line one and
;*	Aarron Vaughan line 2. The S2 switch flips the order.
;*
;*	This is the skeleton file for Lab 4 of ECE 375
;*
;***********************************************************
;*
;*	Author: Bradley Heenk and Aaron Vaughan
;*	Date: 10/30/2019
;*
;***********************************************************

.include "m128def.inc"			; Include definition file

;***********************************************************
;*	Internal Register Definitions and Constants
;***********************************************************
.def	mpr = r16               ; Multipurpose register is required for LCD Driver
.def	countin = r23           ; Counter value for loops
.def	countout = r24          ; Counter value for loops

.equ	forward = 0b11111110    ; Setting up the DP0 switch
.equ	reverse	= 0b11111101    ; Setting up the PD1 switch
.equ	erase = 0b01111111      ; Setting up the PD7 switch

;***********************************************************
;*	Start of Code Segment
;***********************************************************
.cseg                        ; Beginning of code segment

;***********************************************************
;*	Interrupt Vectors
;***********************************************************
.org	$0000                   ; Beginning of IVs
 		rjmp INIT               ; Reset interrupt

.org	$0046                   ; End of Interrupt Vectors

;***********************************************************
;*	Program Initialization
;***********************************************************
INIT:                               ; The initialization routine

		ldi		mpr, low(RAMEND)    ; initialize Stack Pointer
		out		SPL, mpr			
		ldi		mpr, high(RAMEND)
		out		SPH, mpr

		out		DDRD, mpr           ; Set Port D Data Direction Register
		ldi		mpr, $FF            ; Initialize Port D Data Register
		out		PORTD, mpr          ; so all Port D inputs are Tri-State

		ldi		mpr, $00            ; Empyting mpr with zeroes
		ldi		countin, $00        ; Empyting countin with zeroes
		ldi		countout, $00       ; Empyting countout with zeroes

		rcall	LCDInit           ; Initialize LCD Display

		; NOTE that there is no RET or RJMP from INIT, this
		; is because the next instruction executed is the
		; first instruction of the main program

;***********************************************************
;*	Main Program
;***********************************************************
MAIN:
		in		mpr, PIND            ; Record the current values of PIND onto mpr
		cpi		mpr, forward        ; Check to see if forward button is pressed
		brne	CASE1              ; If not check the next case
		rcall	S1_DISPLAY        ; If equal we call S1_DISPLAY
		rjmp	MAIN               ; Start over jump to MAIN
CASE1:	
		cpi		mpr, reverse        ; Check to see if the reverse button is pressed
		brne	CASE2              ; If not check next case
		rcall	S2_DISPLAY        ; If equal we call S2_DISPLAY
		rjmp	MAIN               ; Start over jump to MAIN
CASE2:  
		cpi		mpr, erase          ; Check to see if the erase button is pressed
		brne	MAIN               ; If not jump to MAIN
		rcall	S8_CLEAR          ; If equal we call S8_CLEAR
		rjmp	MAIN               ; Start over jump to MAIN


;***********************************************************
;*	Functions and Subroutines
;***********************************************************

;----------------------------------------------------------------
; Sub:	S1 Display
; Desc:	A micro-function for PARTNER_WRITE to simplify
; what is said in main. This fucntion sets everything up
; to display partner 1 on line 1 and partner 2 on line 2
;----------------------------------------------------------------
S1_DISPLAY:
		push	countin            ; Pushes countin onto the stack
		push	countout           ; Pushes countout onto the stack
		push	mpr                ; Pushes mpr onto the stack

		ldi		countin, $02        ; Declare partner 1 to prep our fucntion
		ldi		countout, $02       ; Declare line 1 to prep our fucntion
		rcall	PARTNER_WRITE     ; Calls PARTNER_WRITE with setup parameters

		ldi		countin, $01        ; Declare partner 2 to prep our fucntion
		ldi		countout, $01       ; Declare line 2 to prep our fucntion
		rcall	PARTNER_WRITE     ; Calls PARTNER_WRITE with setup parameters

		pop		mpr                 ; Pops mpr off the stack
		pop		countout            ; Pops countout off the stack
		pop		countin             ; Pops countin off the stack
ret

;----------------------------------------------------------------
; Sub:	S2 Display
; Desc:	A micro-function for PARTNER_WRITE to simplify
; what is said in main. This fucntion sets everything up
; to display partner 2 on line 1 and partner 1 on line 2
;----------------------------------------------------------------
S2_DISPLAY:
		push	countin             ; Pushes countin onto the stack
		push	countout            ; Pushes countout onto the stack
		push	mpr                 ; Pushes mpr onto the stack

		ldi		countin, $01          ; Declare partner 1 to prep our fucntion
		ldi		countout, $02         ; Declare line 2 to prep our fucntion
		rcall	PARTNER_WRITE       ; Calls PARTNER_WRITE with setup parameters

		ldi		countin, $02          ; Declare partner 2 to prep our fucntion
		ldi		countout, $01         ; Declare line 1 to prep our fucntion
		rcall	PARTNER_WRITE       ; Calls PARTNER_WRITE with setup parameters

		pop		mpr                   ; Pops countin off the stack
		pop		countout              ; Pops countout off the stack
		pop		countin               ; Pops mpr off the stack
ret

;-----------------------------------------------------------
; Func: S8 Clear
; Desc: This fucntion clears the LCD Screen by calling the
; LCDClr in the LCDDriver.asm
;-----------------------------------------------------------
S8_CLEAR:
	rcall	LCDClr              ; Calls the LCDClear function to clear display	
ret

;----------------------------------------------------------------
; Sub:	Writes partner based on the line types
; Desc:	This fucntion depends on countin being set
; before being called to either 1 or a 2 to indicate which 
; partner. The is true with count out correlating to which
; line mpr is set to; 1 means line 1, 2 means line 2.
;----------------------------------------------------------------

PARTNER_WRITE:
		cpi		countin, $01                ; Checks if we're partner 1
		breq	PARTNER1                   ; If true branches to the right Partner1
		cpi		countin, $02                ; Checks if we're partner 2
		breq	PARTNER2                   ; If true branches to the right Partner2
		rjmp	PARTNER_EXIT               ; Countin wasn't setup correctly jump to PARTNER_EXIT
PARTNER1:
		ldi		ZL, low(PARTNER1_BEG<<1)    ; Load left-shifted PARTNER1_BEG low byte into ZL
		ldi		ZH, high(PARTNER1_BEG<<1)  	; Load left-shifted PARTNER1_BEG high byte into ZH
		rjmp	CHECKLINES                 ; Jump to check which line we're on
PARTNER2:
		ldi		ZL, low(PARTNER2_BEG<<1)    ; Load left-shifted PARTNER2_BEG low byte into ZL
		ldi		ZH, high(PARTNER2_BEG<<1)   ; Load left-shifted PARTNER2_BEG high byte into ZH
CHECKLINES:
		cpi		countout, $01               ; Checks if countout is a 1 for line 1
		breq	LINE1                      ; If true branches to LINE1
		cpi		countout, $02               ; Check if countout is a 2 for line 2
		breq	LINE2                      ; If true branches to LINE2
		rjmp	PARTNER_EXIT               ; Countout wasn't setup correctly 
                                ; jump to PARTNER_EXIT
LINE1:
		ldi		YL, low(LCDLn1Addr)         ; Load the LCDLn1Addr into YL low byte
                                ; which points to the beggining of line1
		ldi		YH, high(LCDLn1Addr)        ; Load the LCDLn1Addr into YH high byte
                                ; which points to the beggining of line1
		rjmp	STARTWRITING               ; Jump to STARTWRITING
LINE2:
		ldi		YL, low(LCDLn2Addr)         ; Load the LCDLn2Addr into YL low byte which points
                                ; to the beggining of line2
		ldi		YH, high(LCDLn2Addr)        ; Load the LCDLn2Addr into YH high byte which points
                                ; to the beggining of line2
STARTWRITING:
		ldi		countin, $0F                ; Set out countin counter to start at 16
WRITELINES:
		lpm		mpr, Z+                     ; Load program memory from Z into mpr and post-inc Z
		st		Y+, mpr                      ; Store mpr into Y and post-inc Y
		dec		countin                     ; Decrement our countin
		brne	WRITELINES                 ; If not equal to 0 start loop again

		cpi		countout, $01               ; Checks if we wrote to line 1
		breq	LCDWR1                     ; If true branch to LCDWR1 fucntion
		cpi		countout, $02               ; Checks if we wrote to line 2
		breq	LCDWR2                     ; If true branch to LCDWR2 fucntion
		rjmp	PARTNER_EXIT               ; Countout wasn't setup correctly 
                                ; jump to PARTNER_EXIT
LCDWR1:
		rcall	LCDWrLn1                  ; Call the LCDWrLn1 fucntion to write line 1
		rjmp	PARTNER_EXIT               ; Jump to PARTNER_EXIT
LCDWR2:
		rcall	LCDWrLn2                  ; Call the LCDWrLn2 fucntion to write line 2
PARTNER_EXIT:
ret

;***********************************************************
;*	Stored Program Data
;***********************************************************

;-----------------------------------------------------------
; An example of storing a string. Note the labels before and
; after the .DB directive; these can help to access the data
;-----------------------------------------------------------
PARTNER1_BEG:
.DB		"BRADLEY HEENK   "         ; Declaring data in ProgMemory
PARTNER1_END:

PARTNER2_BEG:
.DB		"AARON VAUGHAN   "         ; Declaring data in ProgMemory
PARTNER2_END:

;***********************************************************
;*	Additional Program Includes
;***********************************************************
.include "LCDDriver.asm"       ; Include the LCD Driver

\end{verbatim}























\section{Challenge Code}

\begin{verbatim}
    ;***********************************************************
;*
;*	Bradley_Heenk_Aaron_Vaughan_Lab4_challenge.asm
;*
;*	This program uses AVR to display strings on the LCD screen
;*	S8 clears the screen, S1 Displays Bradley Heenk line one and
;*	Aarron Vaughan line 2. The S2 switch flips the order.
;*
;*	This is the skeleton file for Lab 4 of ECE 375
;*
;***********************************************************
;*
;*	Author: Bradley Heenk and Aaron Vaughan
;*	Date: 10/31/2019
;*
;***********************************************************

.include "m128def.inc"			; Include definition file

;***********************************************************
;*	Internal Register Definitions and Constants
;***********************************************************
.def	mpr = r16				; Multipurpose register is
								; required for LCD Driver
.def	countin = r23			; Counter value for loops
.def	countout = r24			; Counter value for loops
.def	countwait = r25			; Counter value for loops

.equ	waittime = 25			; Setup wait time for delays
.equ	s1_btn = 0b11111110		; Setting up the S1
.equ	s2_btn	= 0b11111101	; Setting up the S2
.equ	s6_btn = 0b11011111		; Setting up the S8
.equ	s7_btn	= 0b10111111	; Setting up the S7
.equ	s8_btn = 0b01111111		; Setting up the S8

;***********************************************************
;*	Start of Code Segment
;***********************************************************
.cseg							; Beginning of code segment

;***********************************************************
;*	Interrupt Vectors
;***********************************************************
.org	$0000					; Beginning of IVs
 		rjmp INIT				; Reset interrupt

.org	$0046					; End of Interrupt Vectors

;***********************************************************
;*	Program Initialization
;***********************************************************
INIT:								; The initialization routine

		ldi		mpr, low(RAMEND)	; initialize Stack Pointer
		out		SPL, mpr			
		ldi		mpr, high(RAMEND)
		out		SPH, mpr

		out		DDRD, mpr		; Set Port D Data Direction Register
		ldi		mpr, $FF		; Initialize Port D Data Register
		out		PORTD, mpr		; so all Port D inputs are Tri-State

		ldi		mpr, $00		; Empyting mpr with zeroes
		ldi		countin, $00	; Empyting countin with zeroes
		ldi		countout, $00	; Empyting countout with zeroes
		ldi		countwait, $00	; Empyting countwait with zeroes

		rcall	LCDInit			; Initialize LCD Display

		; NOTE that there is no RET or RJMP from INIT, this
		; is because the next instruction executed is the
		; first instruction of the main program

;***********************************************************
;*	Main Program
;***********************************************************
MAIN:
		in		mpr, PIND		; Setup PIND for input
		cpi		mpr, s1_btn		; Compare our first pin to mpr
		brne	CASE1			; If not check the next case
		rcall	S1_INIT			; If equal we call S1_DISPLAY
		rjmp	MAIN			; Start over jump to MAIN
CASE1:	
		cpi		mpr, s2_btn		; Compare reverse pin to mpr
		brne	CASE2			; If not check next case
		rcall	S2_INIT			; If equal we call S2_DISPLAY
		rjmp	MAIN			; Start over jump to MAIN
CASE2:  
		cpi		mpr, s6_btn		; Compare reverse pin to mpr
		brne	CASE3			; If not check next case
		rcall	S6_INIT			; If equal we call S2_DISPLAY
		rjmp	MAIN			; Start over jump to MAIN
CASE3:
		cpi		mpr, s7_btn		; Compare reverse pin to mpr
		brne	CASE4			; If not check next case
		rcall	S7_INIT			; If equal we call S2_DISPLAY
		rjmp	MAIN			; Start over jump to MAIN
CASE4:
		cpi		mpr, s8_btn		; Compare erase pin to mpr
		brne	MAIN			; If not jump to MAIN
		rcall	S8_CLEAR		; If equal we call S8_CLEAR
		rjmp	MAIN			; Start over jump to MAIN


;***********************************************************
;*	Functions and Subroutines
;***********************************************************

;----------------------------------------------------------------
; Sub:	S6 Display
; Desc:	Our main fucntion to have our text scroll from
;		left to right using PD5 where the first line
;		starts scrolling and as characters are cut off
;		they start scrolling from left to right on line 2.
;       For line 2 is also scrolls left to right and as
;		characters are cut off starts scrolling from left
;		to right on line 1 until the names have been swapped.
;----------------------------------------------------------------
S6_DISPLAY:
		ldi		mpr, $00				; Initalize MPR to $00
		ldi		countwait, $01			; Initalize countwait to $01
		ldi		countout, $0F			; Initalize countout to $0F
		ldi		countin, $0F			; Initalize countin to $0F
		
		rcall	S1_INIT					; Calls the S1_INIT function
		rcall	DELAY					; Calls the DELAY function

S1_RELOAD:
		ldi		countout, $11			; Load $11 into countout
		ldi		YL, low(LCDLn1Addr)		; Load the low byte of LCDLn1Addr into YL
		ldi		YH, high(LCDLn1Addr)	; Load the high byte of LCDLn1Addr into YH
		mov		countin, countwait		; Move the value of countwait into countin
S1_L1_SPACELOOP:
		push	mpr							; Push mpr onto the stack
		ldi		ZL, low(PARTNER2_BEG<<1)	; Load the low byte of PARTNER1_BEG into the ZL
		ldi		ZH, high(PARTNER2_BEG<<1)	; Load the high byte of PARTNER1_BEG into the ZH
		ldi		mpr, $10				; Load the value $10 into mpr
		sub		mpr, countin			; Subtract countin from mpr
		add		ZL,	mpr					; Add mpr into ZL
		lpm		mpr, Z+					; Load program memory from Z into mpr and post-inc Z
		st		Y+, mpr					; Store mpr into Y and post increment Y
		pop		mpr						; Pop mpr off the stack

		dec		countin					; Decrement count in by one
		brne	S1_L1_SPACELOOP			; If not equal to zero branch to S1_L1_SPACELOOP
		
		ldi		ZL, low(PARTNER1_BEG<<1)	; Load the low byte of PARTNER1_BEG into ZL
		ldi		ZH, high(PARTNER1_BEG<<1)	; Load the high byte of PARTNER1_BEG into Zh

		ldi		countin, $10			; Load $10 into countin
		sub		countin, countwait		; Subtract countwait from countin

S1_L1_WORDLOOP:
		push	mpr						; Push mpr to the stack
		lpm		mpr, Z+					; Load program memory of Z into mpr and post-inc Z
		st		Y+, mpr					; Store mpr into Y and post increment Y
		pop		mpr						; Pop mpr off the stack
						
		dec		countin					; Decrement countin by one
		brne	S1_L1_WORDLOOP			; If not zero branch to S1_L1_WORDLOOP

		ldi		countout, $11			; Load $11 into countout
		ldi		YL, low(LCDLn2Addr)		; Load the low byte of LCDLn2Addr into YL
		ldi		YH, high(LCDLn2Addr)	; Load the high byte of LCDLn2Addr into YH
		mov		countin, countwait		; Move the value of countwait into countin

		rjmp	S1_L2_SPACELOOP			; Jump to S1_L2_SPACELOOP
UN_JUMP:								; This function was used to allow for large jumps 
										; that require more than 8 bits of memory
		rjmp	S1_RELOAD				; Jump to S1_RELOAD

S1_L2_SPACELOOP:
		push	mpr						; Push mpr onto the stack
		ldi		ZL, low(PARTNER1_BEG<<1)	; Load the low byte of PARTNER1_BEG into ZL
		ldi		ZH, high(PARTNER1_BEG<<1)	; Load the high byte of PARTNER1_BEG into ZH
		ldi		mpr, $10				; Load $10 into mpr
		sub		mpr, countin			; Subtract countin from mpr
		add		ZL,	mpr					; Add mpr to ZL
		lpm		mpr, Z+					; Load from program memory Z into mpr and post-inc Z
		st		Y+, mpr					; Store mpr into Y and post-inc Y
		pop		mpr						; Pop mpr off the stack

		dec		countin					; Decrement countin by one
		brne	S1_L2_SPACELOOP			; If not zero branch to S1_L2_SPACELOOP

		ldi		ZL, low(PARTNER2_BEG<<1)	; Load low byte of PARTNER2_BEG into ZL
		ldi		ZH, high(PARTNER2_BEG<<1)	; Load high byte of PARTNER2_BEG into ZH

		ldi		countin, $10			; Load $10 into countin
		sub		countin, countwait		; Subtract countwait from countin

S1_L2_WORDLOOP:
		push	mpr						; Push mpr onto the stack
		lpm		mpr, Z+					; Load program memory from Z into mpr and post-inc Z
		st		Y+, mpr					; Store mpr into Y and post-inc Y
		pop		mpr						; Pop mpr off the stack

		dec		countin					; Decrement countin by one
		brne	S1_L2_WORDLOOP			; If not zero branch to S1_L2_WORDLOOP

		inc		countwait				; Increment countwait by one

		rcall	DELAY					; Call our DELAY function
		rcall	LCDWrite				; Call LCDWrite function

		cpi		countwait, $00			; Compare countwait to zero
		breq	ALMOSTDONE				; If equal to zero branch to ALSMOSTDONE
		sub		countout, countwait		; Subtract countwait from countin
		brne	UN_JUMP					; If not equal to zero branch to UN_JUMP

ALMOSTDONE:
		rcall	DELAY					; Call the DELAY fucntion
		rcall	S2_INIT					; Call the S2_INIT function

ret

;----------------------------------------------------------------
; Sub:	S7 Display
; Desc:	Our main fucntion to have our text scroll from
;		right to left using PD6 where the first line
;		starts scrolling and as characters are cut off
;		they start scrolling from right to left on line 2.
;       For line 2 is also scrolls right to left and as
;		characters are cut off starts scrolling from right
;		to left on line 1 until the names have been swapped.
;----------------------------------------------------------------
S7_DISPLAY:
		ldi		mpr, $00				; Initalize MPR to $00
		ldi		countwait, $10			; Initalize countwait to $01
		ldi		countout, $0F			; Initalize countout to $0F
		ldi		countin, $0F			; Initalize countin to $0F
		
		rcall	S2_INIT					; Calls the S1_INIT function
		rcall	DELAY					; Calls the DELAY function

S2_RELOAD:
		ldi		countout, $11			; Load $11 into countout
		ldi		YL, low(LCDLn1Addr)		; Load the low byte of LCDLn1Addr into YL
		ldi		YH, high(LCDLn1Addr)	; Load the high byte of LCDLn1Addr into YH
		mov		countin, countwait		; Move the value of countwait into countin
S2_L1_SPACELOOP:
		push	mpr							; Push mpr onto the stack
		ldi		ZL, low(PARTNER2_BEG<<1)	; Load the low byte of PARTNER2_BEG into the ZL
		ldi		ZH, high(PARTNER2_BEG<<1)	; Load the high byte of PARTNER2_BEG into the ZH
		ldi		mpr, $10				; Load the value $10 into mpr
		sub		mpr, countin			; Subtract countin from mpr
		add		ZL,	mpr					; Add mpr into ZL
		lpm		mpr, Z+					; Load program memory from Z into mpr and post-inc Z
		st		Y+, mpr					; Store mpr into Y and post increment Y
		pop		mpr						; Pop mpr off the stack

		dec		countin					; Decrement count in by one
		brne	S2_L1_SPACELOOP			; If not equal to zero branch to S2_L1_SPACELOOP
		
		ldi		ZL, low(PARTNER1_BEG<<1)	; Load the low byte of PARTNER1_BEG into ZL
		ldi		ZH, high(PARTNER1_BEG<<1)	; Load the high byte of PARTNER1_BEG into Zh

		ldi		countin, $10			; Load $10 into countin
		sub		countin, countwait		; Subtract countwait from countin

S2_L1_WORDLOOP:
		push	mpr						; Push mpr to the stack
		lpm		mpr, Z+					; Load program memory of Z into mpr and post-inc Z
		st		Y+, mpr					; Store mpr into Y and post increment Y
		pop		mpr						; Pop mpr off the stack
						
		dec		countin					; Decrement countin by one
		brne	S2_L1_WORDLOOP			; If not zero branch to S2_L1_WORDLOOP

		ldi		countout, $11			; Load $11 into countout
		ldi		YL, low(LCDLn2Addr)		; Load the low byte of LCDLn2Addr into YL
		ldi		YH, high(LCDLn2Addr)	; Load the high byte of LCDLn2Addr into YH
		mov		countin, countwait		; Move the value of countwait into countin

		rjmp	S2_L2_SPACELOOP			; Jump to S2_L2_SPACELOOP
UN_JUMP_2:								; This function was used to allow for large jumps 
										; that require more than 8 bits of memory
		rjmp	S2_RELOAD				; Jump to S2_RELOAD

S2_L2_SPACELOOP:
		push	mpr						; Push mpr onto the stack
		ldi		ZL, low(PARTNER1_BEG<<1)	; Load the low byte of PARTNER1_BEG into ZL
		ldi		ZH, high(PARTNER1_BEG<<1)	; Load the high byte of PARTNER1_BEG into ZH
		ldi		mpr, $10				; Load $10 into mpr
		sub		mpr, countin			; Subtract countin from mpr
		add		ZL,	mpr					; Add mpr to ZL
		lpm		mpr, Z+					; Load from program memory Z into mpr and post-inc Z
		st		Y+, mpr					; Store mpr into Y and post-inc Y
		pop		mpr						; Pop mpr off the stack

		dec		countin					; Decrement countin by one
		brne	S2_L2_SPACELOOP			; If not zero branch to S2_L2_SPACELOOP

		ldi		ZL, low(PARTNER2_BEG<<1)	; Load low byte of PARTNER2_BEG into ZL
		ldi		ZH, high(PARTNER2_BEG<<1)	; Load high byte of PARTNER2_BEG into ZH

		ldi		countin, $10			; Load $10 into countin
		sub		countin, countwait		; Subtract countwait from countin

S2_L2_WORDLOOP:
		push	mpr						; Push mpr onto the stack
		lpm		mpr, Z+					; Load program memory from Z into mpr and post-inc Z
		st		Y+, mpr					; Store mpr into Y and post-inc Y
		pop		mpr						; Pop mpr off the stack

		dec		countin					; Decrement countin by one
		brne	S2_L2_WORDLOOP			; If not zero branch to S2_L2_WORDLOOP

		dec		countwait				; Increment countwait by one

		rcall	DELAY					; Call our DELAY function
		rcall	LCDWrite				; Call LCDWrite function

		cpi		countwait, $00			; Compare countwait to zero
		breq	ALMOSTDONE_2			; If equal to zero branch to ALSMOSTDONE_2
		sub		countout, countwait		; Subtract countwait from countin
		brne	UN_JUMP_2				; If not equal to zero branch to UN_JUMP_2

ALMOSTDONE_2:
		rcall	DELAY					; Call the DELAY function
		rcall	S1_INIT					; Call the S2_INIT function
ret

;----------------------------------------------------------------
; Sub:	S1 Init
; Desc:	A micro-function for PARTNER_WRITE to simplify
;		what is said in main. This fucntion sets everything up
;		to display partner 1 on line 1 and partner 2 on line 2
;----------------------------------------------------------------
S1_INIT:
		push	countin			; Pushes countin onto the stack
		push	countout		; Pushes countout onto the stack
		push	mpr				; Pushes mpr onto the stack

		rcall	LCDClrLn2		; Calls LCD_ClrLn1 function

		ldi		countin, $02	; Declare partner 1 to prep our fucntion
		ldi		countout, $02	; Declare line 1 to prep our fucntion
		rcall	PARTNER_WRITE	; Calls PARTNER_WRITE with setup parameters

		rcall	LCDClrLn1		; Calls LCD_ClrLn2 function

		ldi		countin, $01	; Declare partner 2 to prep our fucntion
		ldi		countout, $01	; Declare line 2 to prep our fucntion
		rcall	PARTNER_WRITE	; Calls PARTNER_WRITE with setup parameters

		pop		mpr				; Pops countin off the stack
		pop		countout		; Pops countout off the stack
		pop		countin			; Pops mpr off the stack
ret

;----------------------------------------------------------------
; Sub:	S2 Init
; Desc:	A micro-function for PARTNER_WRITE to simplify
;		what is said in main. This fucntion sets everything up
;		to display partner 2 on line 1 and partner 1 on line 2
;----------------------------------------------------------------
S2_INIT:
		push	countin			; Pushes countin onto the stack
		push	countout		; Pushes countout onto the stack
		push	mpr				; Pushes mpr onto the stack

		rcall	LCDClrLn2		; Calls LCD_ClrLn1 function

		ldi		countin, $01	; Declare partner 1 to prep our fucntion
		ldi		countout, $02	; Declare line 2 to prep our fucntion
		rcall	PARTNER_WRITE	; Calls PARTNER_WRITE with setup parameters

		rcall	LCDClrLn1		; Calls LCD_ClrLn2 function

		ldi		countin, $02	; Declar partner 2 to prep our fucntion
		ldi		countout, $01	; Declare line 1 to prep our fucntion
		rcall	PARTNER_WRITE	; Calls PARTNER_WRITE with setup parameters

		pop		mpr				; Pops countin off the stack
		pop		countout		; Pops countout off the stack
		pop		countin			; Pops mpr off the stack
ret

;----------------------------------------------------------------
; Sub:	S6 Init
; Desc:	A micro-function for S6_DISPLAY to simplify
;		what is said in main.
;----------------------------------------------------------------
S6_INIT:
		rcall	S6_DISPLAY		; Calls S6_DISPLAY
ret

;----------------------------------------------------------------
; Sub:	S7 Init
; Desc:	A micro-function for S7_DISPLAY to simplify
;		what is said in main.
;----------------------------------------------------------------
S7_INIT:
		rcall	S7_DISPLAY		; Calls S6_DISPLAY
ret

;-----------------------------------------------------------
; Func: S8 Clear
; Desc: This fucntion clears the LCD Screen by calling the
;		LCDClr in the LCDDriver.asm
;-----------------------------------------------------------
S8_CLEAR:
		rcall	LCDClr				; Calls the LCDClear function to clear display	
ret

;----------------------------------------------------------------
; Sub:	Delay
; Desc:	This micro-fucntion takes care of storing our countwait
;		register so we can jsut call this main without having to worry
;----------------------------------------------------------------
DELAY:
		push	countwait				; Pushes countwait to the stack
		ldi		countwait, waittime		; Load our waittime into countwait
		rcall	TIMER					; Call our TIMER function
		pop		countwait				; Pop countwait off the stack
ret

;----------------------------------------------------------------
; Sub:	Wait Function
; Desc:	A wait loop that is 16 + 159975*waitcnt cycles or roughly 
;		waitcnt*10ms.  Just initialize wait for the specific amount 
;		of time in 10ms intervals. Here is the general eqaution
;		for the number of clock cycles in the wait loop:
;		((3 * countin + 3) * countout + 3) * countwait + 13 + call
;----------------------------------------------------------------
TIMER:
		push	countwait			; Save wait register
		push	countin				; Save ilcnt register
		push	countout			; Save olcnt register
		push	mpr

LOOP:	ldi		countout, 224		; load olcnt register
OLOOP:	ldi		countin, 237		; load ilcnt register
ILOOP:	dec		countin				; decrement ilcnt
		brne	ILOOP				; Continue Inner Loop
		dec		countout			; decrement olcnt
		brne	OLOOP				; Continue Outer Loop
		dec		countwait			; Decrement wait 
		brne	LOOP				; Continue Wait loop	

		pop		mpr
		pop		countout			; Restore olcnt register
		pop		countin				; Restore ilcnt register
		pop		countwait			; Restore wait register
ret	

;----------------------------------------------------------------
; Sub:	Writes partner based on the line types
; Desc:	This fucntion depends on countin being set
; before being called to either 1 or a 2 to indicate which 
; partner. The is true with count out which corlates to which
; line mpr is set to; 1 means line 1, 2 means line 2.
;----------------------------------------------------------------
PARTNER_WRITE:
		cpi		countin, $01				; Checks if we're partner 1
		breq	PARTNER1					; If true branches to the right Partner1
		cpi		countin, $02				; Checks if we're partner 2
		breq	PARTNER2					; If true branches to the right Partner2
		rjmp	PARTNER_EXIT				; Countin wasn't setup correctly jump to PARTNER_EXIT
PARTNER1:
		ldi		ZL, low(PARTNER1_BEG<<1)	; Load left-shited PARTNER1_BEG low byte into ZL
		ldi		ZH, high(PARTNER1_BEG<<1)	; Load left-shited PARTNER1_BEG high byte into ZH
		rjmp	CHECKLINES					; Jump to check which line we're on
PARTNER2:
		ldi		ZL, low(PARTNER2_BEG<<1)	; Load left-shited PARTNER2_BEG low byte into ZL
		ldi		ZH, high(PARTNER2_BEG<<1)	; Load left-shited PARTNER2_BEG high byte into ZH
CHECKLINES:
		cpi		countout, $01				; Checks if countout is a 1 for line 1
		breq	LINE1						; If true branches to LINE1
		cpi		countout, $02				; Check if countout is a 2 for line 2
		breq	LINE2						; If true branches to LINE2
		rjmp	PARTNER_EXIT				; Countout wasn't setup correctly jump to PARTNER_EXIT
LINE1:
		ldi		YL, low(LCDLn1Addr)			; Load the LCDLn1Addr into YL low byte which points
											; to the beggining of line1
		ldi		YH, high(LCDLn1Addr)		; Load the LCDLn1Addr into YH high byte which points
											; to the beggining of line1
		rjmp	STARTWRITING				; Jump to STARTWRITING
LINE2:
		ldi		YL, low(LCDLn2Addr)			; Load the LCDLn2Addr into YL low byte which points
											; to the beggining of line2
		ldi		YH, high(LCDLn2Addr)		; Load the LCDLn2Addr into YH high byte which points
											; to the beggining of line2
STARTWRITING:
		ldi		countin, $0F				; Set out countin counter to start at 16
WRITELINES:
		lpm		mpr, Z+						; Load program memory from Z into mpr and post-inc Z
		st		Y+, mpr						; Store mpr into Y and post-inc Y
		dec		countin						; Decrement our countin
		brne	WRITELINES					; If not equal to 0 start loop again

		cpi		countout, $01				; Checks if we wrote to line 1
		breq	LCDWR1						; If true branch to LCDWR1 fucntion
		cpi		countout, $02				; Checks if we wrote to line 2
		breq	LCDWR2						; If true branch to LCDWR2 fucntion
		rjmp	PARTNER_EXIT				; Countout wasn't setup correctly jump to PARTNER_EXIT
LCDWR1:
		rcall	LCDWrLn1					; Call the LCDWrLn1 fucntion to write line 1
		rjmp	PARTNER_EXIT				; Jump to PARTNER_EXIT
LCDWR2:
		rcall	LCDWrLn2					; Call the LCDWrLn2 fucntion to write line 2
PARTNER_EXIT:
ret

;***********************************************************
;*	Additional Program Includes
;***********************************************************
.include "LCDDriver.asm"				; Include the LCD Driver

;***********************************************************
;*	Stored Program Data
;***********************************************************

;-----------------------------------------------------------
; An example of storing a string. Note the labels before and
; after the .DB directive; these can help to access the data
;-----------------------------------------------------------
PARTNER1_BEG:
.DB		"BRADLEY HEENK   "				; Declaring data in ProgMem
PARTNER1_END:

PARTNER2_BEG:
.DB		"AARON VAUGHAN   "				; Declaring data in ProgMem
PARTNER2_END:

\end{verbatim}
\end{document}
