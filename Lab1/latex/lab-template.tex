% template created by: Russell Haering. arr. Joseph Crop
\documentclass[12pt,letterpaper]{article}
\usepackage{anysize}
\marginsize{2cm}{2cm}{1cm}{1cm}

\begin{document}

\begin{titlepage}
    \vspace*{4cm}
    \begin{flushright}
    {\huge
        ECE 375 Lab 1\\[1cm]
    }
    {\large
        Introduction to AVR Tools
    }
    \end{flushright}
    \begin{flushleft}
    Lab Time: Thursday 4-6
    \end{flushleft}
    \begin{flushright}
    Bradley Heenk

    Aaron Vaughan
    \vfill
    \rule{5in}{.5mm}\\
    TA Signature
    \end{flushright}

\end{titlepage}

\section{Introduction}
Text goes here

\section{Internal Register Definitions and Constants}
Text goes here

\section{Interrupt Vectors}
Text goes here

\section{Program Initialization}
Text goes here

\section{Main Program}
Text goes here

\section{A Subroutine}
Text goes here

\section{Stored Program Data}
Text goes here

\section{Additional Questions}
\begin{enumerate}

    \item Go to the lab webpage and download the template write-up. Read it thoroughly and get familiar with the expected format. What specific font is used for source code, and at what size? From here on, when you include your source code in your lab write-up, you must adhere to the specified font type and size.

	\item Go to the lab webpage and read Syllabus carefully. Expected format and
naming convention are very important for submission. If you do not follow naming conventions and formats, you will lose some points. What is the naming convention for source code (asm)? What is the naming convention for source code files if you are working with your partner?
	
	\item Take a look at the code you downloaded for today’s lab. Notice the lines that begin with .def and .equ followed by some type of expression. These are known as pre-compiler directives. Define pre-compiler directive. What is the difference between the .def and .equ directives? (HINT: see Section 5.1 of the AVR Starter Guide).
	
	\item Take another look at the code you downloaded for today’s lab. Read the comment that describes the macro definitions. From that explanation, determine the 8 bit binary value that each of the following expressions evaluates to. Note: the numbers below are decimal values.
	
(a) (1 << 3)
(b) (2 << 2)
(c) (8 >> 1)
(d) (1 << 0)
(e) (6 >> 1|1 << 6)
	
	\item Go to the lab webpage and read the AVR Instruction Set Manual. Based
on this manual, describe the instructions listed below. ADIW, BCLR, BRCC, BRGE, COM, EOR, LSL, LSR, NEG, OR, ORI, ROL, ROR, SBC, SBIW, and SUB.

\end{enumerate}

\section{Difficulties}
Text goes here

\section{Conclusion}
Text goes here

\section{Source Code}
\begin{verbatim}
Source code goes here. It looks best if each line is no
more than 60 characters.
\end{verbatim}
\end{document}
