% template created by: Russell Haering. arr. Joseph Crop
\documentclass[12pt,letterpaper]{article}
\usepackage{anysize}
\marginsize{2cm}{2cm}{1cm}{1cm}

\begin{document}

\begin{titlepage}
    \vspace*{4cm}
    \begin{flushright}
    {\huge
        ECE 375 Lab 1\\[1cm]
    }
    {\large
        Introduction to AVR Tools
    }
    \end{flushright}
    \begin{flushleft}
    Lab Time: Thursday 4-6
    \end{flushleft}
    \begin{flushright}
    Bradley Heenk

    Aaron Vaughan
    \vfill
    \rule{5in}{.5mm}\\
    TA Signature
    \end{flushright}

\end{titlepage}

\section{Introduction}
Text goes here

\section{Internal Register Definitions and Constants}
Text goes here

\section{Interrupt Vectors}
Text goes here

\section{Program Initialization}
Text goes here

\section{Main Program}
Text goes here

\section{A Subroutine}
Text goes here

\section{Stored Program Data}
Text goes here

\section{Additional Questions}
\begin{enumerate}

    \item Go to the lab webpage and download the template write-up. Read it thoroughly and get familiar with the expected format. What specific font is used for source code, and at what size? From here on, when you include your source code in your lab write-up, you must adhere to the specified font type and size.
    
    \begin{enumerate}
    \item The code should be in mono-spaced font and can go down to 8-Pt font to make it fit. When copying the code over it can sometimes get messed up so make sure to clean it up so it looks readable.
    \end{enumerate}

	\item Go to the lab webpage and read Syllabus carefully. Expected format and
naming convention are very important for submission. If you do not follow naming conventions and formats, you will lose some points. What is the naming convention for source code (asm)? What is the naming convention for source code files if you are working with your partner?

	\begin{enumerate}
	\item When naming your .asm file you need to include you and your partners name if you worked together. For example the naming convention needs to be\textbf{"First name\_Last name\_Lab4\_sourcecode (e.g. Bradley\_Heenk\_Lab1\_sourcecode.asm)"}
	\end{enumerate}
	
	\item Take a look at the code you downloaded for today’s lab. Notice the lines that begin with .def and .equ followed by some type of expression. These are known as pre-compiler directives. Define pre-compiler directive. What is the difference between the .def and .equ directives? (HINT: see Section 5.1 of the AVR Starter Guide).
	
	\begin{enumerate}
	\item Pre-compuler directives are defined as a list of special instructions that are executed before the code is even compiled and it directs the compiler. The .DEF is a way to set a symbolic name on a register and the .EQU is to se a symbol equal to an expression. For example we can define .DEF Symbol = Register which can have symbolic names attached to the register to make the coding process easier. On the other hand for the .EQU instructions which assign a value to a label. For example we can run the command .EQU label = expression. This is where a label assigned to a value by the EQU directive and is indeed a constant and cannot be changed.
	\end{enumerate}
	
	\item Take another look at the code you downloaded for today’s lab. Read the comment that describes the macro definitions. From that explanation, determine the 8 bit binary value that each of the following expressions evaluates to. Note: the numbers below are decimal values.

\begin{enumerate}
\item (1 $\ll$ 3)
\item (2 $\ll$ 2)
\item (8 $\gg$ 1)
\item (1 $\ll$ 0)
\item (6 $\gg$ 1$\mid$1 $\ll$ 6)
\end{enumerate}

	\begin{enumerate}
	\item 
	\end{enumerate}
	
	\item Go to the lab webpage and read the AVR Instruction Set Manual. Based
on this manual, describe the instructions listed below. ADIW, BCLR, BRCC, BRGE, COM, EOR, LSL, LSR, NEG, OR, ORI, ROL, ROR, SBC, SBIW, and SUB.

\begin{itemize}
\item ADIW
ADIW falls under the category of Arithmetic and Logic Instructions where these instructions take use of the microcontrollers ALU. AWIW falls under the Addition category based on the table we can see the AD followed by I which is immediate and W which is for the word operation which is a 16-bit operation
\item BCLR
BCLR falls under the category of Bit Manipulation where these instructions allow the programmer to manipulate individual bits within a register. BCLR will set and clear respectively any bit in any I/O register.
\item BRCC
BRCC falls under the category of Conditional Branches this specific case is when the test case $R_{d} \geq R_{r}$ and the boolean is C = 0.
\item BRGE
BRGE also falls under the category of conditional branches and this specific case is when the boolean equation will be $Z+(N \oplus V)=1$.
\item COM
COM falls under the category of arithmetic and logic instructions meaning the compliments.
\item EOR
EOR also falls under the arithmetic and logic instructions under the category of logic
\item LSL
LSL falls under the category of shift and rotate which stands for Logical Shift Left to shirt bits around in a register.
\item LSR
LSR is also under the same category as LSL where bits are shifted around in the register expect LSR means Left Shift right instead of shifting it left.
\item NEG
Also seems to fall under the category of Arithmetic and Logic instructions where it checks to see if the number is negative
\item OR
OR Is under the arithmetic and logical operations meaning to use the OR operator in the ALU.
\item ORI
ORI is also under the arithmetic and logical operations but the I stands for Immediate value that is passed as the second argument
\item ROL
ROL is under the shift and rotate category where we are going to rotate right through carry respectively
\item ROR
ROR is the opposite of ROL where we are going to rotate left through our carry respectively.
\item SBC
SBC falls under the arithmetic and logic instructions and under the category of Subtraction and the C at the end stands with carry.
\item SBIW
SBIW falls under the category of arithmetic and logical instructions under subtraction and the last two letters like the I and W, I stands for an operation that involves an immediate value that is passed as the second argument. While the W stands for the Word value meaning the operation is a 16-bit operation
\item SUB
Falls under the category as the same as SBIW and is the main SUB is for subtraction and is the main basic operation using the ALU in the micro controller.
\end{itemize}

\end{enumerate}

\section{Difficulties}
Text goes here

\section{Conclusion}
Text goes here

\section{Source Code}
\begin{verbatim}
Source code goes here. It looks best if each line is no
more than 60 characters.
\end{verbatim}
\end{document}
