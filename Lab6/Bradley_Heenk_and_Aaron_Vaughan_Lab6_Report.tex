% template created by: Russell Haering. arr. Joseph Crop
\documentclass[12pt, letterpaper]{article}
\usepackage{amssymb,mathtools}
\usepackage[utf8]{inputenc}
\usepackage{setspace}
\setlength{\parskip}{1.2ex}    
\setlength{\parindent}{2em}
\usepackage[a4paper, textwidth=400pt, left=2cm]{geometry}
\renewcommand*\familydefault{\ttdefault} %% Only if the base font of the document is to be typewriter style


\begin{document}


\begin{titlepage}
    \vspace*{4cm}
    \begin{flushright}
    {\huge
        ECE 375 Lab 6\\[1cm]
    }
    {\large
        External Interrupts
    }
    \end{flushright}
    \begin{flushleft}
    Lab Time: Friday 4-6
    \end{flushleft}
    \begin{flushright}
    Aaron Vaughan

    Bradley Heenk
    
    \vfill
    \rule{5in}{.5mm}\\
    TA Signature
    \end{flushright}

\end{titlepage}



\section{Introduction}

The purpose of this lab is to familiarize ourselves with the use of interrupts. We do this by implementing the basic bumpbot code from lab 1 and 2 but instead of using polling to know when to enter a subroutine, we use interrupts. The functionality of the program remains the same with the added complexity of driving the LCD screen to display the number of times that the code executed an interrupt handler function. We use the LCD screen so we need to include the drivers.

%The purpose of this lab is to increase our understanding of the use of the AVR instruction set library to implement common arithmetic operations on 16 to 24-bit unsigned binary numbers. Performing arithmetic operations does not require any additional program includes other than the standard .include "m128def.inc" file. We were tasked with coding a 16-bit adder, 16-bit subtractor, a 24-bit multiplier, and a compound function that uses all three of the other subroutines. The stack pointer is used to jump in and out of subroutine calls. The syntax and organization are becoming second nature by now for my lab partner and I.

\section{Program Overview}

This program implements the normal bumpbot behavior with interrupts to deal with I/O. We had to research the use of interrupts for this lab. Mainly, we needed to know how to control the input hardware on our board and how to ignore all the inputs we didnt need. The buttons on our avr board are active low. For this reason, we initialize our interrupt sense control to activate at the falling edge. To ignore the unneeded inputs we use the EIMSK to mask their values. The rest is the same as our basic bumpbot behavior. The bumpbot moves forward indefinitely. If a whisker is hit, we back up and turn away from that direction and resume moving forward. There is a counter display on the LCD screen this lab. Each time an interrupt is triggered the counter on the LCD will increment. We do this to keep track of the number of times each whisker is hit. There is a function to clear the left and right whisker counter on the LCD as well.

%This program performs addition, subtraction, multiplication and a compound function that uses all three operations.
%The setup is basic, in that we define some register names and call instructions out of the AVR instruction set. The tricky part is when jumping in and out of the subroutine calls, we must push all the registers onto the stack to preserve their values for the main program. Then, when exiting we must pop them back off the stack in reverse order, thus preserving their values. We must allocate some SRAM space to hold our input and output variables for each function. After hard coding the values into the SRAM during simulation, all of the functions can be called one after the other. The data space allocated to the output will be updated upon completion of each of the subroutines and displayed in the memory window in the debugger simulation.

\section{Internal Register Definitions and Constants}
 The LCD screen uses R17 to R22 so we are limited in our use of multipurpose registers. We set up R16 as mpr R23 as waitcnt, R24 and R25 are our loop counters. We designate the right and left whisker as constant values that will be used to select the corresponding input button. 

% Unlike the last lab, we do not have a driver include that sucks up all of the register resources. We had complete freedom to use as many registers as we wanted to implement the subroutines. We defined 6 registers that would be used for performing arithmetic operations on our variables. This configuration allowed us to hold entire 24-bit values in an easily recognizable form. For the use of multiplication function, we needed to use R0, and R1, but in the end, we never actually use the "MUL" instruction. There are two loop counters used. There is a constant value register named "zero" that may have been useful to clear a register value or variable in the SRAM data memory. 

\section{Interrupt Vectors}

Of course we have the reset interrupt at \$0000. We also use four other interrupt vectors to handle the subroutine execution: \$0002 is HitLeft, \$0004 is HitRight, \$0006 is HitLeftClr which clears the HitLeft counter, and \$0008 is the HitRightClr vector. These vectors just call the subroutine and return back to the location of main before the interrupt was initiated.

%There is one interrupt vector within the skeleton code for this lab. It sets up the initial starting point of our main program. Beginning at memory address \$0000 it simply jumps to the initialization routine.

\section{Program Initialization}

For this lab we need the stack pointer initialized. We use it in the LCD driver file and for the subroutines to push and pop our registers to save the state of the processor. We need PORTB as an output bus for use with the LEDs and PORTD as an input bus for the push buttons. As described above, the buttons are acitive low. We set up the interrupt control registers EICRA and EICRB to activate on a falling edge. Next we set up the EIMSK to mask the pins that we do not want to monitor. We then initialize the LCD screen by simply calling the function provided by the LCDDriver.asm file. We then call a reset function to set the lcd output counters to zero. Lastly, we set the global interrupt bit.

%The zero register is set to zero in this section. Also the stack pointer must be set up in order to use the push \& pop instructions.

%The stack pointer must be initialized to use the stack in the algorithm. This is typically done, as implemented in the our code, to the end of ram. It takes two cycles to perform this because the stack pointer is 16-bits but the register used to communicate with it is only 8-bits wide. Without the use of the stack pointer and use of that type of data structure we would lose track of the main program memory location.

\section{Main Program}

This part is simple. Just move the TekBot forward indefinitely. The interrupts account for all of the subroutine calls.

%The main program runs through each of the subroutine calls with a "NOP" instruction on either side of it to halt the simulation for use by the graders to check the otput/input values. When each of the subroutines are called, the main program runs in an infinite loop.

%Before going into each of the subroutine calls we set up the values of the operands to be used by that subroutine by setting the Z pointer to the location in program memory that holds the value. We then use the LPM instruction to move it to a register and then store it out to the data space memory.



\section{Subroutines}

Subroutine Name: HitLeft \newline
Description: Handles functionality of the TekBot when the right whisker	is triggered. \newline

Subroutine Name: HitRight \newline
Description: Handles functionality of the TekBot when the right whisker is triggered. \newline

Function Name: Hit Right Clear \newline
Description: Prepares our update fucntion to be set to a "0" then calls our fucntion to update the screen this is used for when we want to clear a specific button that is pressed \newline

Function Name: Hit Left Clear \newline
Description: Prepares our update fucntion to be set to a "0" then calls our fucntion to update the screen this is used for when we want to clear a specific button that is pressed. \newline

Subroutine Name: Wait \newline
Description: A wait loop that is 16 + 159975*waitcnt cycles or roughly waitcnt*10ms.  Just initialize wait for the specific amount of time in 10ms intervals. Here is the general eqaution for the number of clock cycles in the wait loop: ((3 * ilcnt + 3) * olcnt + 3) * waitcnt + 13 + call \newline

Function Name: Queue Fix Function \newline
Description: This fucntions counts to around 600 micro seconds this is used to help avoid queue delays since we know our Atmega128 chip runs at 16 mhz we know each clock cycle will be 1 / 16 mhz and convert it to microseconds. Now we can take 600 value and divide by our result which was used to determine how many loops of 255 clock cycles we would need. \newline

Subroutine Name: Corner Case \newline
Description: This function checks alternating left and right whisker pushes on our board if it alternates left then right for total of 5 times it will turn 180 degrees and move forward instead \newline

Function Name: Im Stuck \newline
Description: This function is called if the TekBot is stuck in a corner and turns 180 degrees to get unstuck from the corner. \newline

Function Name: SameWhisker \newline
Description: This function is called when the same whisker is hit repeatedly. It extends the turning and backing up routine by one second to get more clearance from an obstacle that is repeatedly in our way. \newline

Function Name: Update Left Function \newline
Description: This uses the value in olcnt and loads that char into the LCDscreen and updates that specific char \newline

Function Name: Update Right Function \newline
Description: This uses the value in ilcnt and loads that char into the LCDscreen and updates that specific char. \newline

Function Name: Base Text Function \newline
Description: This uses the value in ilcnt and loads that char into the LCDscreen and updates that specific char \newline



\section{Stored Program Data}

We allocate some space in program memory to display a label on the LCD screen to signify which counter is being incremented. They aer named "LW: " and "RW: " for left and right whisker respectively.

%We allocate program memory to store the values of our operands.

\section{Additional Program Includes}

The LCD driver file must be included in this lab to utilize the functionality of the LCD screen. "LCDDriver.asm"


\section{Additional Questions}
\begin{enumerate}
    \item
As this lab, Lab 1, and Lab 2 have demonstrated, there are always multiple ways to accomplish the same task when programming (this is especially true for assembly programming). As an engineer, you will need to be able to justify your design choices. You have now seen the BumpBot behavior implemented using two different programming languages (AVR assembly and C), and also using two different methods of receiving external input (polling and interrupts). Explain the benefits and costs of each of these approaches. Some important areas of interest include, but are not limited to: efficiency, speed, cost of context switching, programming time, understandability, etc. \newline \newline

The main benefit of the use of interrupts is the understandability of the code. The main program is much simpler. The use of interrupts limits the lines of code in main by handling the calling and returning of each of the subroutine cases. The use of the LCD screen complicated the code considerably but before this section was added, the code is very streamlined. Above all, the C-language programming is the most understandable because it is performed in a high-level programming language. The high-level tactic comes with its own set of complications though. If we are concerned with the speed and precision of each operation the C-Language may hinder our ability to control \textit{exactly} what the processor is doing. For this reason, the risk of unforeseen bugs using a high-level coding language offers some area of concern.
When considering the time to program, the polling and the interrupt coding methods were the about equal. The C-Language was fastest but I think that is because of my familiarity to that language.


\item
Instead of using the Wait function that was provided in \newline BasicBumpBot.asm, is it possible to use a timer/counter interrupt to perform the one-second delays that are a part of the BumpBot behavior, while still using external interrupts for the bumpers? Give a reasonable argument either way, and be sure to mention if interrupt priority had any effect on your answer.

Since using the timer/counterx in ctc mode would not interfere with the interrupt pins, we would be able to use it to perform the wait sequence.


\end{enumerate}

\section{Difficulties}

Since the TA's forgot to tell us to remove the jumper J11 or J10 we got stuck on the clear function for HOURS!! We commented out the code and the errors still existed. This made no sense. We emailed one of the TAs and they told us to remove the jumper and this completely fixed our problem.

\section{Conclusion}

Seeing how interrupts work was interesting to us. The code was much more streamlined using this technique rather than pin polling. The polling technique was much more confusing to read and understand. We spent hours on the clear function before one of the TAs told us to remove a jumper. The interrupt INT3 and INT2 pin must have some connectivity to J10 and J11 on the board. This is an interesting fact that I will not soon forget. Implementing the basic bumpbot functionality was extremely simple. We coded, compiled, uploaded the hexfile to the avr board and it worked flawlessly. We spent the next three days trying to make the interrupt counter on the LCD work. After coding we compiled and uploaded it to the board and tested. The clear function was clearing both interrupt counters. After removing the jumper, and retesting, the code did what it was supposed to do. 

We then took care of the challenge code. I sure wish we had some more registers to work with. This part was extremely challenging. Providing some real world solutions to the TekBot getting stuck was rewarding. 

% In this lab, we were required to implement some basic arithmetic functions using AVR assembly code. The lab was largely uninteresting and tedious. We coded the project, compiled, then debugged for a bit, then set up the break points that will be used by the TA's to check the functionality of our subroutines. One useful thing that I learned was how to hard code values into the data memory from program memory.

\section{Source Code \& Challenge Code}

\begin{verbatim}

;***********************************************************
;*
;*	Bradley_Heenk_and_Aaron_Vaughan_Lab6_challenegecode.asm
;*
;*	This program uses the process or interrupts intstead of
;*	polling for out BumpBots it also displays on the screen
;*	how many times each of the left or right whiskers are hit
;*	this displays up to a maximum of 19 which was how it was
;*	designed which is more than enough for this lab. This
;*	program also detects when we're int a corner or hit
;*	the same whisker twice.
;*
;*	This is the skeleton file for Lab 6 of ECE 375
;*
;***********************************************************
;*
;*	 Author: Bradley Heenk and Aaron Vaughan
;*	   Date: 11/13/2019
;*
;***********************************************************

.include "m128def.inc"			; Include definition file

;***********************************************************
;*	Internal Register Definitions and Constants
;***********************************************************
.def	mpr = r16				; Multipurpose register 
.def	waitcnt = r23
.def	ilcnt = r24
.def	olcnt = r25
.def	memory = r5
.def	lastwsk = r2
.def	leftcnt = r3
.def	rightcnt = r4
.def	currentwsk = r6
.def	checkbit = r7

.equ	WskrR = 0				; Right Whisker Input Bit
.equ	WskrL = 1				; Left Whisker Input Bit

;***********************************************************
;*	Start of Code Segment
;***********************************************************
.cseg							; Beginning of code segment

;***********************************************************
;*	Interrupt Vectors
;***********************************************************
.org	$0000					; Beginning of IVs
		rjmp 	INIT			; Reset interrupt

.org	$0002					; Beginning of IVs
		rcall 	HitRight		; Reset interrupt for HitLeft
		reti

.org	$0004
		rcall 	HitLeft			; Reset interrupt for HitRight
		reti

.org	$0006
		rcall 	HitRightClr		; Reset interrupt for HitLeftClr
		reti

.org	$0008
		rcall 	HitLeftClr		; Reset interrupt for HitRightClr
		reti

.org	$0046					; End of Interrupt Vectors

;***********************************************************
;*	Program Initialization
;***********************************************************
INIT:	; The initialization routine
		; Initialize Stack Pointer
		ldi		mpr, low(RAMEND)
		out		SPL, mpr
		ldi		mpr, high(RAMEND)
		out		SPH, mpr

		; Initialize Port B for out output
		ldi		mpr, $00
		out		PORTB, mpr
		ldi		mpr, $FF
		out		DDRB, mpr

		; Initialize Port D for our inputs
		ldi		mpr, $FF
		out		PORTD, mpr
		ldi		mpr, $00
		out		DDRD, mpr

		; Initialize external interrupts
		ldi	mpr, 0b10101010			; Set the Interrupt Sense Control to falling edge 
		sts	EICRA, mpr
		
		ldi	mpr, 0b00000000			; Set the Interrupt Sense Control to falling edge 
		out	EICRB, mpr

		; Configure the External Interrupt Mask
		ldi	mpr, 0b00001111			; Set value to what we want to hide
		out	EIMSK, mpr

		rcall	LCDInit			; Initialize LCD Display

		; Get our screen intialized to zero to start with
		rcall	HitLeftClr
		rcall	HitRightClr

		; Turn on interrupts
		sei
			; NOTE: This must be the last thing to do in the INIT function

;***********************************************************
;*	Main Program
;***********************************************************
MAIN:							; The Main program

		; Turns off the motors 
		ldi	mpr,	0xF0		; Load 0xF0 into mpr to indicate stopped
		out	PORTB,	mpr			; Store mpr into the I/O register of PORTB

		; Move the bumpbot forward
		ldi	mpr,	0x60		; Load 0x60 into mpr to move forward
		out	PORTB,	mpr			; Store mpr into the I/O register of PORTB

		rjmp	MAIN			; Create an infinite while loop to signify the 
								; end of the program.

;***********************************************************
;*	Functions and Subroutines
;***********************************************************

;----------------------------------------------------------------
; Sub:	HitLeft
; Desc:	Handles functionality of the TekBot when the right whisker
;		is triggered.
;----------------------------------------------------------------
HitLeft:							; Begin a function with a label

		; Save variable by pushing them to the stack
		push	mpr
		push	waitcnt
		in  	mpr, SREG
		push	mpr

		ldi		mpr, $01			; Load 1 into mpr
		mov		currentwsk, mpr		; Load mpr into the current whisker

		; Setup our UpdateRight function for 0
		push	olcnt
		mov		olcnt, leftcnt		; Move our current left whisker counter to olcnt
		rcall	UpdateLeft			; Call our update counter function
		pop		olcnt

		rcall	CornerCase			; Call the cornercase function to check if we are stuck
		sbrs	checkbit,0			; Check if the checkbit is set and if so skip rjmp LEFTSKIP
		rjmp	LEFTSKIP			; Relative jump to LEFTSKIP

		; Execute the function here
		; Preform reverse command wait 100 ms
		ldi		mpr,	0x0
		out		PORTB,	mpr
		ldi		waitcnt, 100
		rcall	WAITFUNC

		; Preform left command wait 100 ms
		ldi		mpr,	0x20
		out		PORTB,	mpr
		ldi		waitcnt, 100
		rcall	WAITFUNC

		; Preform forward command
		ldi		mpr,	0x60
		out		PORTB,	mpr

LEFTSKIP:

		rcall	QueueFix			; Call the QueueFix function for 600 us delay
		ldi		mpr, $03			; Load $03 into mpr and have a 
		out		EIFR, mpr

		ldi		mpr, $01			; Load $01 into mpr
		mov		lastwsk, mpr		; Load mpr into the lastwsk

		ldi		mpr, $01			; Load $01 into mpr
		mov		checkbit, mpr		; Load mpr into the checkbit

		; Restore variable by popping them from the stack in reverse order
		pop 	mpr
		out 	SREG, mpr
		pop 	waitcnt
		pop 	mpr

		ret						; End a function with RET

;----------------------------------------------------------------
; Sub:	HitRight
; Desc:	Handles functionality of the TekBot when the right whisker
;		is triggered.
;----------------------------------------------------------------
HitRight:						; Begin a function with a label

		; Save variable by pushing them to the stack
		push	mpr	
		push	waitcnt
		in  	mpr, SREG
		push	mpr

		ldi		mpr, $02
		mov		currentwsk, mpr

		; Setup our UpdateRight function for 0
		push	ilcnt
		mov		ilcnt, rightcnt
		rcall	UpdateRight			; Call the update right function 
		pop		ilcnt

		rcall	CornerCase			; Call our corner case function
		sbrs	checkbit,0			; Skip the next step if the bit if checkbit is set
		rjmp	RIGHTSKIP			; Jump to RIGTHSKIP

		; Execute the function here
		; Preform reverse command wait 100 ms
		ldi		mpr,	0x0
		out		PORTB,	mpr
		ldi		waitcnt, 100
		rcall	WAITFUNC

		; Preform right command wait 100 ms
		ldi		mpr,	0x40
		out		PORTB,	mpr
		ldi		waitcnt, 100
		rcall	WAITFUNC

		; Preform forward command
		ldi		mpr,	0x60
		out		PORTB,	mpr

RIGHTSKIP:

		rcall	QueueFix			; Call the queue function for 600 us delay
		ldi		mpr, $03			; Store $03 into mpr
		out		EIFR, mpr			; Load mpr into the I/O of EIFR

		ldi		mpr, $02			; $02 into mpr
		mov		lastwsk, mpr		; Load mpr nto lastwsk
			
		ldi		mpr, $01			; Load $01 into mpr
		mov		checkbit, mpr		; Load 1 into checkbit

		; Restore variable by popping them from the stack in reverse order
		pop 	mpr
		out 	SREG, mpr
		pop 	waitcnt
		pop 	mpr

		ret						; End a function with RET

;-----------------------------------------------------------
; Func: Hit Right Clear Function
; Desc: Prepares our update fucntion to be set to a "0"
;		then calls our fucntion to update the screen
;		this is used for when we want to clear a specific
;		button that is pressed
;-----------------------------------------------------------
HitRightClr:						; Begin a function with a label

		push	mpr				; Push registers onto the stack

		ldi		mpr, $30				; Load the value of $30 into mpr
		mov		rightcnt, mpr			; Load mpr into rightcnt
		ldi		mpr, $00				; Load 0 into mpr
		mov		lastwsk, mpr			; Set the last wsk to 0
		mov		currentwsk, mpr			; Set the current wsk to 0
		mov		memory, mpr				; Set the memory to 0

		; Setup our UpdateLeft function
		push	ilcnt
		mov		ilcnt, rightcnt	; Load $30 ("0") into olcnt
		rcall	UpdateRight		; Call UpdateRight
		pop		ilcnt

		rcall	QueueFix		; Call the QueueFix function
		ldi		mpr, $03		; Load value of 3 into mpr
		out		EIFR, mpr		; Store to the I/O of EIFR with mpr

		pop		mpr				; Pop registers onto the stack

		ret						; End a function with RET		

;-----------------------------------------------------------
; Func: Hit Clear Left
; Desc: Prepares our update fucntion to be set to a "0"
;		then calls our fucntion to update the screen
;		this is used for when we want to clear a specific
;		button that is pressed
;-----------------------------------------------------------
HitLeftClr:						; Begin a function with a label
		
		push	mpr				; Push registers onto the stack

		ldi		mpr, $30				; Load the value $30 into mpr
		mov		leftcnt, mpr			; Load leftcnt with mpr
		ldi		mpr, $00				; Load the value of 0 into mpr
		mov		lastwsk, mpr			; Set the last wsk to 0
		mov		currentwsk, mpr			; Set the current wsk to 0
		mov		memory, mpr				; Set the memory to 0

		; Setup our UpdateLeft function
		push	olcnt
		mov		olcnt, leftcnt		; Load $30 ("0") into olcnt
		rcall	UpdateLeft		; Call UpdateLeft
		pop		olcnt

		rcall	QueueFix		; Call the QueueFix function
		ldi		mpr, $03		; Load value of 3 into mpr
		out		EIFR, mpr		; Store to the I/O of EIFR with mpr

		pop		mpr				; Pop registers off the stack

		ret						; End a function with RET	

;----------------------------------------------------------------
; Sub:	Wait
; Desc:	A wait loop that is 16 + 159975*waitcnt cycles or roughly 
;		waitcnt*10ms.  Just initialize wait for the specific amount 
;		of time in 10ms intervals. Here is the general eqaution
;		for the number of clock cycles in the wait loop:
;			((3 * ilcnt + 3) * olcnt + 3) * waitcnt + 13 + call
;----------------------------------------------------------------
WAITFUNC:
		push	waitcnt			; Save wait register
		push	ilcnt			; Save ilcnt register
		push	olcnt			; Save olcnt register

Loop:	ldi		olcnt, 224		; load olcnt register
OLoop:	ldi		ilcnt, 237		; load ilcnt register
ILoop:	dec		ilcnt			; decrement ilcnt
		brne	ILoop			; Continue Inner Loop
		dec		olcnt			; decrement olcnt
		brne	OLoop			; Continue Outer Loop
		dec		waitcnt			; Decrement wait 
		brne	Loop			; Continue Wait loop	

		pop		olcnt			; Restore olcnt register
		pop		ilcnt			; Restore ilcnt register
		pop		waitcnt			; Restore wait register
		ret						; Return from subroutine

;-----------------------------------------------------------
; Func: Queue Fix Function
; Desc: This fucntions counts to around 600 micro seconds
;		this is used to help avoid queue delays since we
;		know our Atmega128 chip runs at 16 mhz we know each
;		clock cycle will be 1 / 16 mhz and convert it to
;		microseconds. Now we can take 600 value and divide
;		by out result which was used to determine how many
;		loops of 255 clock cycles we would need to stack
;		hence the inner loop and outer loops
;-----------------------------------------------------------
QueueFix:
		push	ilcnt			; Push registers onto the stack
		push	olcnt
		ldi		ilcnt, 255		; Load 255 into ilcnt
		ldi		olcnt, 1		; Load 30 into olcnt
ILOOPQUEUE:
		dec		ilcnt			; Decrement ilcnt
		brne	ILOOPQUEUE		; Branch if not equal to zero to ILOOPQUEUE
OLOOPQUEUE:
		ldi		ilcnt, 255		; Load 255 into ilcnt
		dec		olcnt			; Decrement olcnt
		brne	ILOOPQUEUE		; Branch if not equal to zero to ILOOPQUEUE

		pop		olcnt			; Pop registers off the stack
		pop		ilcnt
ret

;----------------------------------------------------------------
; Sub:	Corner Case
; Desc:	This function checks alternating left and right whisker
;		pushes on our board if it alternates left then right for
;		a total of 5 times it will turn 180 degrees and move
;		forward instaed
;----------------------------------------------------------------
CornerCase:

		push	mpr

		mov		mpr, lastwsk		; Move laskwsk into mpr

		cpi		mpr, 0				; Compare mpr to 0
		breq	CORNEREXIT			; If true brance to CORNEREXIT
		mov		mpr, currentwsk		; move currentwsk into mpr
		cpi		mpr, 1				; Compare mpr to 1
		breq	LEFTWHISKER			; If true branch to LEFTWHISKER
		cpi		mpr, 2				; Compare mpr to 2
		breq	RIGHTWHISKER		; If true branch to RIGHTWHISKER
		rjmp	CORNEREXIT			; Sanity check in case mpr is garbage


LEFTWHISKER:	; If left whisker is hit this is executed
		cp		currentwsk, lastwsk ; Compare current whisker to last hisker
		breq	SAME				; If true we hit the same whisker and branch to SAME
		
		mov		mpr, memory			; Load our memory for our cases into mpr
		andi	mpr, $F0			; And the 4 most significant bits and store into mpr

		cpi		mpr, 0b01000000		; Check if the left most bits = 4
		breq	STUCK				; If true execute our STUCK corner case function

		push	olcnt				; Push olcnt onto the stack
		ldi		olcnt, $10			; Load $01 into olcnt
		add		mpr, olcnt			; add olcnt to mpr and store into mpr
		
		mov		olcnt, memory		; Move our memory into olcnt
		andi	olcnt, $0F			; And olcnt with 0F getting us the right most significant bits
		add		mpr, olcnt			; Add olcnt with mpr and store intro
		pop		olcnt 				; Pop olcnt off the stack

		mov		memory, mpr			; move mpr into memory and replace it

		rjmp	CORNEREXIT			; Go to the end of our function

RIGHTWHISKER:	; If right whisker is hit this is executed
		cp		currentwsk, lastwsk ; Compare last whisker to current whisker
		breq	SAME				; If true branch to the same meaning the same whisker is hit

		mov		mpr, memory			; Load our memory into mpr
		andi	mpr, $0F			; And the 4 least significant bits and store into mpr

		cpi		mpr, 0b00000100		; Compare the least signicant 4 bits
		breq	STUCK				; if true branch to STUCK statement

		push	olcnt				; Push olcnt onto the stack
		ldi		olcnt, $01			; Load $01 into olcnt
		add		mpr, olcnt			; Add mpr with olcnt and store into mpr
		
		mov		olcnt, memory		; Move our memory into olcnt
		andi	olcnt, $F0			; And the 4 most signficant bits and store into olcnt
		add		mpr, olcnt			; Add olcnt with mpr and store into mpr
		pop		olcnt 				; Pop olcnt off the stack

		mov		memory, mpr			; Move our new mpr value into memory

		rjmp	CORNEREXIT		; Go to the end of our function

SAME:
		rcall	SameWhisker			; We ended up in this area call SameWhisker
		rjmp	CORNEREXIT			; Jump to the end of our function

STUCK:
		rcall	ImStuck				; We ended up in this area call ImStuck
		rjmp	CORNEREXIT			; Jump to the end of our function

CORNEREXIT:
		
		pop		mpr					; Pop mpr off the stack

		ret						; Return from subroutine

;-----------------------------------------------------------
; Func: Im Stuck
; Desc: This function preforms the flip and turn 180 degrees
;		to turn around and get unstuck from the corner
;-----------------------------------------------------------
ImStuck:
		ldi		mpr, $00		; Load $00 into mpr
		mov		memory, mpr		; Move mpr into memory to reset our memory

		; Execute the function here
		; Preform reverse command wait 100 ms
		ldi		mpr,	0x0
		out		PORTB,	mpr
		ldi		waitcnt, 100
		rcall	WAITFUNC

		; Preform right command wait 100 ms
		ldi		mpr,	0x40
		out		PORTB,	mpr

		; Wait for 4 seconds
		ldi		waitcnt, 200
		rcall	WAITFUNC
		ldi		waitcnt, 200
		rcall	WAITFUNC

		; Preform forward command
		ldi		mpr,	0x60
		out		PORTB,	mpr

		ldi		mpr, $00		; Load $00 into mpr
		mov		checkbit, mpr	; Move mpr into the checkbit

		ret

;-----------------------------------------------------------
; Func: Same Whisker
; Desc: This fucntion calls the same whisker fucntion
;		when the same whisker is hit to avoid hitting
;		the same object over and over again.
;-----------------------------------------------------------
SameWhisker:
		ldi		mpr, $00		; Load $00 into mpr
		mov		memory, mpr		; Move mpr into memory to reset our memory

		; Execute the function here
		; Preform reverse command wait 100 ms
		ldi		mpr,	0x0
		out		PORTB,	mpr
		ldi		waitcnt, 200
		rcall	WAITFUNC

		; Preform right command wait 100 ms
		ldi		mpr,	0x40
		out		PORTB,	mpr

		; Wait for 2 seconds
		ldi		waitcnt, 200
		rcall	WAITFUNC
		dec		ilcnt

		; Preform forward command
		ldi		mpr,	0x60
		out		PORTB,	mpr

		ldi		mpr, $00		; Load $00 into mpr
		mov		checkbit, mpr	; Move mpr into the checkbit

		ret
;-----------------------------------------------------------
; Func: Update Left Function
; Desc: This uses the value in olcnt and loads that char
;		into the LCDscreen and updates that specific char
;-----------------------------------------------------------
UpdateLeft:						; Begin a function with a label

		push	mpr

		rcall	BaseText

		ldi		YL, low(LCDLn1Addr)		; Load the low byte of LCDLn1Addr into YL
		ldi		YH, high(LCDLn1Addr)	; Load the high byte of LCDLn1Addr into YH


		ldi		mpr, 4					; Load 14 into mpr this used for where on the screen we want to be
		add		YL, mpr					; Now add this value to YL

		mov		mpr, olcnt				; Copy and store olcnt into mpr
		cpi		mpr, $3A				; Compare mpr to $3A which is greater then 9			
		brge	TENSLEFT				; This indicated if were above 9 

		st		Y+, mpr					; Store mpr into Y
		
		ldi		mpr, $20				; Load $20 into mpr, $20 => " " character
		st		Y, mpr					; Store this value into Y

		rjmp	LEFTDONE				; End the fucntion by calling RIGHTDONE

TENSLEFT:
		ldi		mpr, $31				; Load a 1 into our MSB in our display
		st		Y+, mpr					; Store this value in our display

		mov		mpr, olcnt				; Copy and store olcnt into mpr
		subi	mpr, 10
		st		Y, mpr					; Store mpr into Y

LEFTDONE:

		rcall	LCDWrLn1				; Call the LCDWrite function to update the display

		inc		leftcnt

		pop		mpr

ret						; End a function with RET

;-----------------------------------------------------------
; Func: Update Right Function
; Desc: This uses the value in ilcnt and loads that char
;		into the LCDscreen and updates that specific char
;-----------------------------------------------------------
UpdateRight:						; Begin a function with a label

		push	mpr

		rcall	BaseText

		ldi		YL, low(LCDLn1Addr)		; Load the low byte of LCDLn1Addr into YL
		ldi		YH, high(LCDLn1Addr)	; Load the high byte of LCDLn1Addr into YH

		ldi		mpr, 14					; Load 14 into mpr this used for where on the screen we want to be
		add		YL, mpr					; Now add this value to YL

		mov		mpr, ilcnt				; Copy and store ilcnt into mpr
		cpi		mpr, $3A				; Compare mpr to $3A which is greater then 9			
		brge	TENSRIGHT				; This indicated if were above 9 

		st		Y+, mpr					; Store mpr into Y
		
		ldi		mpr, $20				; Load $20 into mpr, $20 => " " character
		st		Y, mpr					; Store this value into Y

		rjmp	RIGHTDONE				; End the fucntion by calling RIGHTDONE

TENSRIGHT:
		ldi		mpr, $31				; Load a 1 into our MSB in our display
		st		Y+, mpr					; Store this value in our display

		mov		mpr, ilcnt				; Copy and store ilcnt into mpr
		subi	mpr, 10
		st		Y, mpr					; Store mpr into Y

RIGHTDONE:

		rcall	LCDWrLn1				; Call the LCDWrite function to update the display

		inc		rightcnt				; Increment rightcnt

		pop		mpr

ret						; End a function with RET


;-----------------------------------------------------------
; Func: Base Text Function
; Desc: This uses the value in ilcnt and loads that char
;		into the LCDscreen and updates that specific char
;-----------------------------------------------------------
BaseText:						; Begin a function with a label

		push	mpr

		ldi		YL, low(LCDLn1Addr)		; Load the low byte of LCDLn1Addr into YL
		ldi		YH, high(LCDLn1Addr)	; Load the high byte of LCDLn1Addr into YH
		ldi		ZL, low(LW_BEG<<1)
		ldi		ZH, high(LW_BEG<<1)

		lpm		mpr, Z+					; Load program memory from where Z points into mpr
		st		Y+, mpr					; Store mpr into Y and post inc
		lpm		mpr, Z+					; Load program memory from where Z points into mpr
		st		Y+, mpr					; Store mpr into Y and post inc
		lpm		mpr, Z+					; Load program memory from where Z points into mpr
		st		Y+, mpr					; Store mpr into Y and post inc
		lpm		mpr, Z+					; Load program memory from where Z points into mpr
		st		Y+, mpr					; Store mpr into Y and post inc

		ldi		mpr, 6
		add		YL, mpr

		ldi		ZL, low(RW_BEG<<1)
		ldi		ZH, high(RW_BEG<<1)

		lpm		mpr, Z+					; Load program memory from where Z points into mpr
		st		Y+, mpr					; Store mpr into Y and post inc
		lpm		mpr, Z+					; Load program memory from where Z points into mpr
		st		Y+, mpr					; Store mpr into Y and post inc
		lpm		mpr, Z+					; Load program memory from where Z points into mpr
		st		Y+, mpr					; Store mpr into Y and post inc
		lpm		mpr, Z+					; Load program memory from where Z points into mpr
		st		Y+, mpr					; Store mpr into Y and post inc

		pop		mpr

ret						; End a function with RET

;***********************************************************
;*	Stored Program Data
;***********************************************************

LW_BEG:
.DB		"LW: "		; Declaring data in ProgMem
LW_END:

RW_BEG:
.DB		"RW: "		; Declaring data in ProgMem
RW_END:

; Enter any stored data you might need here

;***********************************************************
;*	Additional Program Includes
;***********************************************************
; There are no additional file includes for this program
.include "LCDDriver.asm"				; Include the LCD Driver



\end{verbatim}

\end{document}
